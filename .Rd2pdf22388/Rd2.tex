\nonstopmode{}
\documentclass[a4paper]{book}
\usepackage[times,inconsolata,hyper]{Rd}
\usepackage{makeidx}
\usepackage[utf8]{inputenc} % @SET ENCODING@
% \usepackage{graphicx} % @USE GRAPHICX@
\makeindex{}
\begin{document}
\chapter*{}
\begin{center}
{\textbf{\huge Package `thresholding'}}
\par\bigskip{\large \today}
\end{center}
\inputencoding{utf8}
\ifthenelse{\boolean{Rd@use@hyper}}{\hypersetup{pdftitle = {thresholding: Langston Lab Graph Thresholding Tools}}}{}
\ifthenelse{\boolean{Rd@use@hyper}}{\hypersetup{pdfauthor = {Gabriel Johnson; Carissa Bleker}}}{}
\begin{description}
\raggedright{}
\item[Type]\AsIs{Package}
\item[Title]\AsIs{Langston Lab Graph Thresholding Tools}
\item[Version]\AsIs{1.0.0}
\item[Date]\AsIs{2023-10-17}
\item[Encoding]\AsIs{UTF-8}
\item[Maintainer]\AsIs{Levi Hochstetler }\email{lhochste@vols.utk.edu}\AsIs{}
\item[Description]\AsIs{This package encapsulates several graph thresholding utilities and analysis tools. 
Thresholding a graph helps remove lesser important edges from a graph, and potentially 
shrinking the graph by pruning off disconnected vertices. The analysis tools take in a graph input file, 
and the user specifies what operations to perform on the graph to help determine the optimal threshold 
for the graph. These thresholding tools rely on the igraph C library for graph creation and manipulation. 
Other functionalities are used from the alglib mathematical library as well. }
\item[URL]\AsIs{}\url{https://github.com/carissableker/thresholding}\AsIs{}
\item[License]\AsIs{GPL (>= 2)}
\item[Imports]\AsIs{Rcpp (>= 1.0.11), ggplot2, dplyr, magrittr, stringr, pracma}
\item[Depends]\AsIs{Rcpp (>= 1.0.11)}
\item[LinkingTo]\AsIs{Rcpp}
\item[RoxygenNote]\AsIs{7.2.3}
\item[NeedsCompilation]\AsIs{yes}
\item[Author]\AsIs{Levi Hochstetler [cre],
Gabriel Johnson [aut],
Carissa Bleker [aut]}
\end{description}
\Rdcontents{\R{} topics documented:}
\inputencoding{utf8}
\HeaderA{thresholding-package}{A short title line describing what the package does}{thresholding.Rdash.package}
\aliasA{thresholding}{thresholding-package}{thresholding}
\keyword{package}{thresholding-package}
%
\begin{Description}
A more detailed description of what the package does. A length
of about one to five lines is recommended.
\end{Description}
%
\begin{Details}
This section should provide a more detailed overview of how to use the
package, including the most important functions.
\end{Details}
%
\begin{Author}
Your Name, email optional.

Maintainer: Your Name <your@email.com>
\end{Author}
%
\begin{References}
This optional section can contain literature or other references for
background information.
\end{References}
%
\begin{SeeAlso}
Optional links to other man pages
\end{SeeAlso}
%
\begin{Examples}
\begin{ExampleCode}
  ## Not run: 
     ## Optional simple examples of the most important functions
     ## These can be in \dontrun{} and \donttest{} blocks.   
  
## End(Not run)
\end{ExampleCode}
\end{Examples}
\inputencoding{utf8}
\HeaderA{analysis}{Main graph thresholding analysis function}{analysis}
%
\begin{Description}
Main graph thresholding analysis function
\end{Description}
%
\begin{Usage}
\begin{verbatim}
analysis(
  infile,
  outfile_prefix,
  methods = "",
  lower = 0.5,
  upper = 0.99,
  increment = 0.01,
  window_size = 5L,
  min_partition_size = 10L,
  min_clique_size = 5L,
  min_alpha = 0,
  max_alpha = 4,
  alpha_increment = 0.1,
  num_samples = 0L,
  significance_alpha = 0.01,
  bonferroni_corrected = 0L
)
\end{verbatim}
\end{Usage}
%
\begin{Arguments}
\begin{ldescription}
\item[\code{infile:}] Name of .ncol graph file to read in for analysis

\item[\code{outfile\_prefix:}] Prefix of output file in which analysis will be redirected to (Ex: <PREFIX>.iterative.txt )

\item[\code{methods:}] Comma separated list of analysis methods, listed if thresholding::help() is called (defaults to none)
\end{ldescription}
\end{Arguments}
\inputencoding{utf8}
\HeaderA{edge\_hist}{edge\_hist}{edge.Rul.hist}
%
\begin{Description}
Histogram function for displaying edge weight frequencies
\end{Description}
%
\begin{Usage}
\begin{verbatim}
edge_hist(infile, bin_width = 0.01, sep = "\t")
\end{verbatim}
\end{Usage}
%
\begin{Arguments}
\begin{ldescription}
\item[\code{infile}] tab separated .ncol formatted graph file (list of weighted edges - wel)

\item[\code{bin\_width}] width of each histogram bin (default is 0.01)

\item[\code{sep}] determines how the wel file is separated (default is tab separated file)
\end{ldescription}
\end{Arguments}
\inputencoding{utf8}
\HeaderA{function\_test}{Manually exported in NAMESPACE}{function.Rul.test}
%
\begin{Description}
Manually exported in NAMESPACE
\end{Description}
%
\begin{Usage}
\begin{verbatim}
function_test(c)
\end{verbatim}
\end{Usage}
\inputencoding{utf8}
\HeaderA{get\_iterative\_t\_values}{Translated from Carissa Bleker's Thresholding helper functions for quicker analysis of thresholding analysis results}{get.Rul.iterative.Rul.t.Rul.values}
%
\begin{Description}
The link to Carissa's Github repo: 
https://github.com/carissableker/thresholding
The link to the combine\_analysis\_results functions: 
https://github.com/carissableker/thresholding/blob/master/example/combine\_analysis\_results.py
Helper function for thresholding::get\_results()
Gets thresholding values from the iterative results
Reads in tab the separated .iterative.txt files from 
thresholding::analysis
\end{Description}
%
\begin{Usage}
\begin{verbatim}
get_iterative_t_values(files, D = NULL, d_min_t = list(general = 0))
\end{verbatim}
\end{Usage}
%
\begin{Arguments}
\begin{ldescription}
\item[\code{files}] Vector of file paths or just one file path

\item[\code{D}] Optional argument that accepts a list. This can be used
to capture the thresholding results by a certain method. If just
using get\_iter\_t\_vals, then D can be left unspecified. Otherwise,
get\_results fills out and returns this value automatically

\item[\code{d\_min\_t}] INTERNAL CONTROL
\end{ldescription}
\end{Arguments}
\inputencoding{utf8}
\HeaderA{get\_iter\_t\_vals}{get\_iter\_t\_vals() User wrapper function for get\_iterative\_t\_values Returns the thresholding data frame created by the internal get\_iterative\_t\_values, which does not return anything to the user.}{get.Rul.iter.Rul.t.Rul.vals}
%
\begin{Description}
The returned data frame includes graph and graph method values for each 
increment of the threshold.
\end{Description}
%
\begin{Usage}
\begin{verbatim}
get_iter_t_vals(outfile_prefix)
\end{verbatim}
\end{Usage}
%
\begin{Arguments}
\begin{ldescription}
\item[\code{outfile\_prefix}] Prefix of output file, which can have several
output file paths if the same prefix is run several. This assumes
that the desired file(s) are in the current working directory
(pwd)

Ex.) get\_iter\_t\_vals("iter-prefix") will execute get\_iterative\_t\_values
on all files names iter-prefix-\#\#\#.iterative.txt where \#\#\# is the process
ID of the internal process (like 4318, 3341, 414143, etc.).
\end{ldescription}
\end{Arguments}
\inputencoding{utf8}
\HeaderA{get\_results}{Prints the resulting analysis method thresholding values after running thresholding::analysis()}{get.Rul.results}
%
\begin{Description}
Note: This function assumes that the .iterative anaylsis() output file
is in the current directory unless the path to the file is provided.
\end{Description}
%
\begin{Usage}
\begin{verbatim}
get_results(outfile_prefix, plot_iterative = FALSE)
\end{verbatim}
\end{Usage}
%
\begin{Arguments}
\begin{ldescription}
\item[\code{outfile\_prefix}] filename or file path for resulting output file from 
running the analysis function (file would be <prefix>.iterative.txt)
\end{ldescription}
\end{Arguments}
\inputencoding{utf8}
\HeaderA{help}{Display argument information on terminal for thresholding::analysis()}{help}
%
\begin{Description}
Display argument information on terminal for thresholding::analysis()
\end{Description}
%
\begin{Usage}
\begin{verbatim}
help()
\end{verbatim}
\end{Usage}
\inputencoding{utf8}
\HeaderA{plot\_t\_vs\_ev}{Plots the edge (E) and vertex (V)count of a graph at varying thresholds Marks non-null methods and their optimal thresholds against the  V/E line plots}{plot.Rul.t.Rul.vs.Rul.ev}
%
\begin{Description}
Plots the edge (E) and vertex (V)count of a graph at varying thresholds
Marks non-null methods and their optimal thresholds against the 
V/E line plots
\end{Description}
%
\begin{Usage}
\begin{verbatim}
plot_t_vs_ev(plot_df, D)
\end{verbatim}
\end{Usage}
%
\begin{Arguments}
\begin{ldescription}
\item[\code{plot\_df}] Dataframe with threshold values - returned from 
output of get\_iterative\_results, which contains the detailed
analysis of the graph at each increment of the thresholding
process

\item[\code{D}] List of methods and their optimal thresholds from 
calling get\_results. The user can either pass the resulting variable
from get\_results or the list itself (i.e. D=variable\$D instead of D=variable).
\end{ldescription}
\end{Arguments}
\inputencoding{utf8}
\HeaderA{print\_vector}{Prints out 10 numbers}{print.Rul.vector}
%
\begin{Description}
Prints out 10 numbers
\end{Description}
%
\begin{Usage}
\begin{verbatim}
print_vector(i)
\end{verbatim}
\end{Usage}
%
\begin{Arguments}
\begin{ldescription}
\item[\code{i}] - an integer that gets printed right back out
Manually exported in NAMESPACE
\end{ldescription}
\end{Arguments}
\inputencoding{utf8}
\HeaderA{rcpp\_hello\_world}{Simple function using Rcpp}{rcpp.Rul.hello.Rul.world}
%
\begin{Description}
Simple function using Rcpp
\end{Description}
%
\begin{Usage}
\begin{verbatim}
rcpp_hello_world()	
\end{verbatim}
\end{Usage}
%
\begin{Examples}
\begin{ExampleCode}
## Not run: 
rcpp_hello_world()

## End(Not run)
\end{ExampleCode}
\end{Examples}
\printindex{}
\end{document}
